\documentclass[]{article}
\usepackage{lmodern}
\usepackage{amssymb,amsmath}
\usepackage{ifxetex,ifluatex}
\usepackage{fixltx2e} % provides \textsubscript
\ifnum 0\ifxetex 1\fi\ifluatex 1\fi=0 % if pdftex
  \usepackage[T1]{fontenc}
  \usepackage[utf8]{inputenc}
\else % if luatex or xelatex
  \ifxetex
    \usepackage{mathspec}
  \else
    \usepackage{fontspec}
  \fi
  \defaultfontfeatures{Ligatures=TeX,Scale=MatchLowercase}
\fi
% use upquote if available, for straight quotes in verbatim environments
\IfFileExists{upquote.sty}{\usepackage{upquote}}{}
% use microtype if available
\IfFileExists{microtype.sty}{%
\usepackage{microtype}
\UseMicrotypeSet[protrusion]{basicmath} % disable protrusion for tt fonts
}{}
\usepackage[margin=1in]{geometry}
\usepackage{hyperref}
\hypersetup{unicode=true,
            pdftitle={Taller No. 2},
            pdfauthor={Kevin Pelaez - Juan Diego Osorio},
            pdfborder={0 0 0},
            breaklinks=true}
\urlstyle{same}  % don't use monospace font for urls
\usepackage{color}
\usepackage{fancyvrb}
\newcommand{\VerbBar}{|}
\newcommand{\VERB}{\Verb[commandchars=\\\{\}]}
\DefineVerbatimEnvironment{Highlighting}{Verbatim}{commandchars=\\\{\}}
% Add ',fontsize=\small' for more characters per line
\usepackage{framed}
\definecolor{shadecolor}{RGB}{248,248,248}
\newenvironment{Shaded}{\begin{snugshade}}{\end{snugshade}}
\newcommand{\KeywordTok}[1]{\textcolor[rgb]{0.13,0.29,0.53}{\textbf{#1}}}
\newcommand{\DataTypeTok}[1]{\textcolor[rgb]{0.13,0.29,0.53}{#1}}
\newcommand{\DecValTok}[1]{\textcolor[rgb]{0.00,0.00,0.81}{#1}}
\newcommand{\BaseNTok}[1]{\textcolor[rgb]{0.00,0.00,0.81}{#1}}
\newcommand{\FloatTok}[1]{\textcolor[rgb]{0.00,0.00,0.81}{#1}}
\newcommand{\ConstantTok}[1]{\textcolor[rgb]{0.00,0.00,0.00}{#1}}
\newcommand{\CharTok}[1]{\textcolor[rgb]{0.31,0.60,0.02}{#1}}
\newcommand{\SpecialCharTok}[1]{\textcolor[rgb]{0.00,0.00,0.00}{#1}}
\newcommand{\StringTok}[1]{\textcolor[rgb]{0.31,0.60,0.02}{#1}}
\newcommand{\VerbatimStringTok}[1]{\textcolor[rgb]{0.31,0.60,0.02}{#1}}
\newcommand{\SpecialStringTok}[1]{\textcolor[rgb]{0.31,0.60,0.02}{#1}}
\newcommand{\ImportTok}[1]{#1}
\newcommand{\CommentTok}[1]{\textcolor[rgb]{0.56,0.35,0.01}{\textit{#1}}}
\newcommand{\DocumentationTok}[1]{\textcolor[rgb]{0.56,0.35,0.01}{\textbf{\textit{#1}}}}
\newcommand{\AnnotationTok}[1]{\textcolor[rgb]{0.56,0.35,0.01}{\textbf{\textit{#1}}}}
\newcommand{\CommentVarTok}[1]{\textcolor[rgb]{0.56,0.35,0.01}{\textbf{\textit{#1}}}}
\newcommand{\OtherTok}[1]{\textcolor[rgb]{0.56,0.35,0.01}{#1}}
\newcommand{\FunctionTok}[1]{\textcolor[rgb]{0.00,0.00,0.00}{#1}}
\newcommand{\VariableTok}[1]{\textcolor[rgb]{0.00,0.00,0.00}{#1}}
\newcommand{\ControlFlowTok}[1]{\textcolor[rgb]{0.13,0.29,0.53}{\textbf{#1}}}
\newcommand{\OperatorTok}[1]{\textcolor[rgb]{0.81,0.36,0.00}{\textbf{#1}}}
\newcommand{\BuiltInTok}[1]{#1}
\newcommand{\ExtensionTok}[1]{#1}
\newcommand{\PreprocessorTok}[1]{\textcolor[rgb]{0.56,0.35,0.01}{\textit{#1}}}
\newcommand{\AttributeTok}[1]{\textcolor[rgb]{0.77,0.63,0.00}{#1}}
\newcommand{\RegionMarkerTok}[1]{#1}
\newcommand{\InformationTok}[1]{\textcolor[rgb]{0.56,0.35,0.01}{\textbf{\textit{#1}}}}
\newcommand{\WarningTok}[1]{\textcolor[rgb]{0.56,0.35,0.01}{\textbf{\textit{#1}}}}
\newcommand{\AlertTok}[1]{\textcolor[rgb]{0.94,0.16,0.16}{#1}}
\newcommand{\ErrorTok}[1]{\textcolor[rgb]{0.64,0.00,0.00}{\textbf{#1}}}
\newcommand{\NormalTok}[1]{#1}
\usepackage{graphicx,grffile}
\makeatletter
\def\maxwidth{\ifdim\Gin@nat@width>\linewidth\linewidth\else\Gin@nat@width\fi}
\def\maxheight{\ifdim\Gin@nat@height>\textheight\textheight\else\Gin@nat@height\fi}
\makeatother
% Scale images if necessary, so that they will not overflow the page
% margins by default, and it is still possible to overwrite the defaults
% using explicit options in \includegraphics[width, height, ...]{}
\setkeys{Gin}{width=\maxwidth,height=\maxheight,keepaspectratio}
\IfFileExists{parskip.sty}{%
\usepackage{parskip}
}{% else
\setlength{\parindent}{0pt}
\setlength{\parskip}{6pt plus 2pt minus 1pt}
}
\setlength{\emergencystretch}{3em}  % prevent overfull lines
\providecommand{\tightlist}{%
  \setlength{\itemsep}{0pt}\setlength{\parskip}{0pt}}
\setcounter{secnumdepth}{0}
% Redefines (sub)paragraphs to behave more like sections
\ifx\paragraph\undefined\else
\let\oldparagraph\paragraph
\renewcommand{\paragraph}[1]{\oldparagraph{#1}\mbox{}}
\fi
\ifx\subparagraph\undefined\else
\let\oldsubparagraph\subparagraph
\renewcommand{\subparagraph}[1]{\oldsubparagraph{#1}\mbox{}}
\fi

%%% Use protect on footnotes to avoid problems with footnotes in titles
\let\rmarkdownfootnote\footnote%
\def\footnote{\protect\rmarkdownfootnote}

%%% Change title format to be more compact
\usepackage{titling}

% Create subtitle command for use in maketitle
\newcommand{\subtitle}[1]{
  \posttitle{
    \begin{center}\large#1\end{center}
    }
}

\setlength{\droptitle}{-2em}

  \title{Taller No. 2}
    \pretitle{\vspace{\droptitle}\centering\huge}
  \posttitle{\par}
    \author{Kevin Pelaez - Juan Diego Osorio}
    \preauthor{\centering\large\emph}
  \postauthor{\par}
      \predate{\centering\large\emph}
  \postdate{\par}
    \date{28 de octubre de 2018}


\begin{document}
\maketitle

\begin{verbatim}
## 
## Attaching package: 'PolynomF'
\end{verbatim}

\begin{verbatim}
## The following object is masked from 'package:pracma':
## 
##     integral
\end{verbatim}

\subsection{PUNTO \#1}\label{punto-1}

\begin{enumerate}
\def\labelenumi{\arabic{enumi}.}
\tightlist
\item
  Considere un cuerpo con temperatura interna \textbf{T}𝑇 el cual se
  encuentra en un ambiente con temperatura constante\textbf{Te}. Suponga
  que su masa \textbf{m}𝑚 concentrada en un solo punto. Entonces la
  transferencia de calor entre el cuerpo y el entorno externo puede ser
  descrita con la ley de Stefan-Boltzmann:
\end{enumerate}

\[
v(t) = εγS(T^4(t)-T_e^4) 
\] Donde, \textbf{t} es tiempo y \textbf{ε} es la constante de Boltzmann
\((ε = 5.6x10^-8 J/m^2K^2s)\), \textbf{γ} es la constante de
``Emisividad'' del cuerpo, \textbf{S} el área de superficie y \textbf{v}
es la tasa de transferencia de calor. La tasa de variacion de la energía
\(dT/dt = -v(t)/mC\) (C indica el calor específico del material que
constituye el cuerpo). En consecuencia,

\[
dT/dt = -εγS(T^4(t)-T_e^4)/mC
\] Usando el método de Euler (en R) y 20 intervalos iguales y t variando
de 0 a 200 segundos, resuelva numéricamente la ecuación, si el cuerpo es
un cubo de lados de longitud 1m y masa igual a 1Kg. Asuma, que T0 =
180K, Te = 200K, g = 0.5 y C = 100J/(Kg/K). Hacer una representación
gráfica del resultado.

\begin{verbatim}
##      X        Y
## 21 200 192.6369
\end{verbatim}

\includegraphics{DocProyecto_files/figure-latex/unnamed-chunk-2-1.pdf}

\subsection{PUNTO \#2}\label{punto-2}

\begin{enumerate}
\def\labelenumi{\arabic{enumi}.}
\setcounter{enumi}{1}
\tightlist
\item
  Obtenga cinco puntos de la solución de la ecuación, utilizando el
  método de Taylor (los tres primeros términos)con h=0.1 implemente en R
\end{enumerate}

\[
    dy/dx - (x+y) = 1 - x^2; y(0) =1 
\]

Grafique su solución y compare con la solución exacta, cuál es el error
de truncamiento en cada paso

\begin{Shaded}
\begin{Highlighting}[]
\NormalTok{funcion4punto <-}\StringTok{ }\ControlFlowTok{function}\NormalTok{(x)\{}
  \KeywordTok{exp}\NormalTok{(x) }\OperatorTok{*}\StringTok{ }\NormalTok{(x}\OperatorTok{^}\DecValTok{2}\OperatorTok{*}\StringTok{ }\KeywordTok{exp}\NormalTok{(}\OperatorTok{-}\NormalTok{x) }\OperatorTok{+}\StringTok{ }\NormalTok{x}\OperatorTok{*}\StringTok{ }\KeywordTok{exp}\NormalTok{(}\OperatorTok{-}\NormalTok{x) }\OperatorTok{+}\StringTok{ }\DecValTok{1}\NormalTok{)}
\NormalTok{\}}

\NormalTok{num <-}\StringTok{ }\KeywordTok{seq}\NormalTok{(}\DecValTok{1}\OperatorTok{:}\DecValTok{4}\NormalTok{)}
\NormalTok{plotsish <-}\StringTok{ }\KeywordTok{c}\NormalTok{(}\KeywordTok{taylor}\NormalTok{(funcion4punto, }\DataTypeTok{x0=}\DecValTok{0}\NormalTok{, }\DataTypeTok{n =} \DecValTok{3}\NormalTok{))}

\KeywordTok{plot}\NormalTok{(plotsish, }\DataTypeTok{col =} \StringTok{"darkgreen"}\NormalTok{, }\DataTypeTok{main =} \StringTok{"TAYLOR"}\NormalTok{)}

\NormalTok{poliajuste <-}\StringTok{ }\KeywordTok{poly.calc}\NormalTok{(num,plotsish)}
\KeywordTok{curve}\NormalTok{(poliajuste,}\DataTypeTok{add =} \OtherTok{TRUE}\NormalTok{, }\DataTypeTok{col =} \StringTok{"yellow"}\NormalTok{)}
\end{Highlighting}
\end{Shaded}

\includegraphics{DocProyecto_files/figure-latex/unnamed-chunk-3-1.pdf}

\subsection{PUNTO \#3}\label{punto-3}

\begin{enumerate}
\def\labelenumi{\arabic{enumi}.}
\setcounter{enumi}{2}
\tightlist
\item
  Obtenga 20 puntos de la solución de la ecuación, utilizando el método
  de Euler (los tres primeros términos) con h=0.1 \[
  dy/dx - (x+y) = 1 - x^2; y(0) =1 
  \] Grafique su solución y compare con la solución exacta, cuál es el
  error de truncamiento en cada paso
\end{enumerate}

\subsection{PUNTO \#4}\label{punto-4}

Implemente en R el siguiente algoritmo y aplíquelo para resolver la
ecuación anterior

\begin{enumerate}
\def\labelenumi{\arabic{enumi})}
\tightlist
\item
  Defina h y la cantidad de puntos a calcular m
\item
  Defina f(x,y) y la condicion iniccial (x0,y0)
\item
  para i =12, \ldots{}, m
\item
  K1= hf(xi, yi)
\item
  k2 = hf(xi + h, yi +h)
\item
  yi+1 = yi + 1/2 (k1 + k2)
\item
  xi+1 = xi + h
\item
  fin
\end{enumerate}

\begin{Shaded}
\begin{Highlighting}[]
\NormalTok{funcion4punto <-}\StringTok{ }\ControlFlowTok{function}\NormalTok{(x,y)\{}
  \KeywordTok{return}\NormalTok{  (}\KeywordTok{exp}\NormalTok{(x) }\OperatorTok{*}\StringTok{ }\NormalTok{(x}\OperatorTok{^}\DecValTok{2}\OperatorTok{*}\StringTok{ }\KeywordTok{exp}\NormalTok{(}\OperatorTok{-}\NormalTok{x) }\OperatorTok{+}\StringTok{ }\NormalTok{x}\OperatorTok{*}\StringTok{ }\KeywordTok{exp}\NormalTok{(}\OperatorTok{-}\NormalTok{x) }\OperatorTok{+}\StringTok{ }\DecValTok{1}\NormalTok{))}
\NormalTok{\}}

\NormalTok{m <-}\StringTok{ }\DecValTok{5}
  
\NormalTok{h <-}\StringTok{ }\FloatTok{0.1}
\NormalTok{ x0 <-}\StringTok{ }\DecValTok{1}
\NormalTok{ y0 <-}\StringTok{ }\DecValTok{0}
\ControlFlowTok{for}\NormalTok{ (i  }\ControlFlowTok{in} \DecValTok{1}\OperatorTok{:}\NormalTok{m)\{}
\NormalTok{  k1 <-}\StringTok{ }\NormalTok{h }\OperatorTok{*}\StringTok{ }\KeywordTok{funcion4punto}\NormalTok{(x0, y0)}
\NormalTok{  k2 <-}\StringTok{ }\NormalTok{h }\OperatorTok{*}\StringTok{ }\KeywordTok{funcion4punto}\NormalTok{(x0 }\OperatorTok{+}\StringTok{ }\NormalTok{h, y0}\OperatorTok{+}\NormalTok{h)}
\NormalTok{  y0 <-}\StringTok{ }\NormalTok{y0 }\OperatorTok{+}\StringTok{ }\FloatTok{0.5} \OperatorTok{*}\StringTok{ }\NormalTok{(k1 }\OperatorTok{+}\StringTok{ }\NormalTok{k2)}
\NormalTok{  x0 <-}\StringTok{ }\NormalTok{x0 }\OperatorTok{+}\StringTok{ }\NormalTok{h}
\NormalTok{ \}}
\end{Highlighting}
\end{Shaded}

\subsection{PUNTO \#5}\label{punto-5}

Utilizar la siguiente variación en el método de Euler, para resolver una
ecuación diferencial ordinaria de primer orden, la cual calcula el
promedio de las pendientes en cada paso

\$\$

y\_i+\{1 =\} y\_i + h/2 f((x\_i,y\_i) + f(x\_i+\{1, \}y\_i+\{1) \})

\$\$

Implemente un código en R, para este método y obtenga 10 puntos de la
solución con h=0.1, grafíquela y compárela con el método de Euler:

\[
    dy/dx - (x+y) = 1 - x^2 =0 ; y(0) =1 
\]

\begin{Shaded}
\begin{Highlighting}[]
\NormalTok{variacionMetodoEuler <-}\StringTok{ }\ControlFlowTok{function}\NormalTok{(f, h, xi, yi, xf)}
\NormalTok{\{}
\NormalTok{  N =}\StringTok{ }\NormalTok{(xf }\OperatorTok{-}\StringTok{ }\NormalTok{xi) }\OperatorTok{/}\StringTok{ }\NormalTok{h}
\NormalTok{  x =}\StringTok{ }\NormalTok{y =}\StringTok{ }\KeywordTok{numeric}\NormalTok{(N}\OperatorTok{+}\DecValTok{1}\NormalTok{)}
\NormalTok{  x[}\DecValTok{1}\NormalTok{] =}\StringTok{ }\NormalTok{xi; }
\NormalTok{  y[}\DecValTok{1}\NormalTok{] =}\StringTok{ }\NormalTok{yi;}
\NormalTok{  i =}\StringTok{ }\DecValTok{1}
  \ControlFlowTok{while}\NormalTok{ (i }\OperatorTok{<=}\StringTok{ }\NormalTok{N)}
\NormalTok{  \{}
\NormalTok{    x[i}\OperatorTok{+}\DecValTok{1}\NormalTok{] =}\StringTok{ }\NormalTok{x[i]}\OperatorTok{+}\NormalTok{h}
\NormalTok{    y[i}\OperatorTok{+}\DecValTok{1}\NormalTok{] =}\StringTok{ }\NormalTok{y[i]}\OperatorTok{+}\NormalTok{(h}\OperatorTok{/}\DecValTok{2}\NormalTok{)}\OperatorTok{*}\NormalTok{(}\KeywordTok{f}\NormalTok{(x[i],y[i]))}
\NormalTok{    i =}\StringTok{ }\NormalTok{i}\OperatorTok{+}\DecValTok{1}
\NormalTok{  \}}
  \KeywordTok{return}\NormalTok{ (}\KeywordTok{data.frame}\NormalTok{(}\DataTypeTok{X =}\NormalTok{ x, }\DataTypeTok{Y =}\NormalTok{ y))}
\NormalTok{\}}
\NormalTok{f <-}\StringTok{ }\ControlFlowTok{function}\NormalTok{(x,y) \{x}\OperatorTok{+}\NormalTok{y}\OperatorTok{-}\DecValTok{1}\OperatorTok{+}\NormalTok{x}\OperatorTok{^}\DecValTok{2}\NormalTok{\}}

\NormalTok{e1 =}\StringTok{ }\KeywordTok{variacionMetodoEuler}\NormalTok{(f, }\FloatTok{0.1}\NormalTok{, }\DecValTok{0}\NormalTok{, }\DecValTok{1}\NormalTok{, }\DecValTok{1}\NormalTok{)}

\NormalTok{e1[}\KeywordTok{nrow}\NormalTok{(e1),]}
\end{Highlighting}
\end{Shaded}

\begin{verbatim}
##    X        Y
## 11 1 1.414725
\end{verbatim}

\begin{Shaded}
\begin{Highlighting}[]
\KeywordTok{par}\NormalTok{(}\DataTypeTok{mfrow =} \KeywordTok{c}\NormalTok{(}\DecValTok{1}\NormalTok{,}\DecValTok{2}\NormalTok{))}

\NormalTok{xx <-}\StringTok{ }\KeywordTok{c}\NormalTok{(}\OperatorTok{-}\DecValTok{3}\NormalTok{, }\DecValTok{3}\NormalTok{); yy <-}\StringTok{ }\KeywordTok{c}\NormalTok{(}\OperatorTok{-}\DecValTok{1}\NormalTok{, }\DecValTok{1}\NormalTok{)}
\KeywordTok{vectorfield}\NormalTok{(f, xx, yy, }\DataTypeTok{scale =} \FloatTok{0.1}\NormalTok{)}
\ControlFlowTok{for}\NormalTok{ (xs }\ControlFlowTok{in} \KeywordTok{seq}\NormalTok{(}\OperatorTok{-}\DecValTok{1}\NormalTok{, }\DecValTok{1}\NormalTok{, }\DataTypeTok{by =} \FloatTok{0.25}\NormalTok{)) }
\NormalTok{\{}
\NormalTok{  sol <-}\StringTok{ }\KeywordTok{rk4}\NormalTok{(f, }\OperatorTok{-}\DecValTok{1}\NormalTok{, }\DecValTok{1}\NormalTok{, xs, }\DecValTok{100}\NormalTok{)}
  \KeywordTok{lines}\NormalTok{(sol}\OperatorTok{$}\NormalTok{x, sol}\OperatorTok{$}\NormalTok{y, }\DataTypeTok{col=}\StringTok{"purple"}\NormalTok{)}
\NormalTok{\}}
\KeywordTok{title}\NormalTok{(}\DataTypeTok{main=}\StringTok{"Campo Vectorial"}\NormalTok{, }\DataTypeTok{col.main=}\StringTok{"black"}\NormalTok{, }\DataTypeTok{font.main=}\DecValTok{4}\NormalTok{)}

\KeywordTok{plot}\NormalTok{(e1, }\DataTypeTok{col =} \StringTok{"darkblue"}\NormalTok{, }\DataTypeTok{main =} \StringTok{"Grafica"}\NormalTok{)}
\end{Highlighting}
\end{Shaded}

\includegraphics{DocProyecto_files/figure-latex/unnamed-chunk-5-1.pdf}

\subsection{PUNTO \#7}\label{punto-7}

Pruebe el siguiente código en R del método de Runge Kutta de tercer y
cuarto orden y obtenga 10 puntos de la solución con h=0.1, grafíquela y
compárela con el método de Euler:

\[
    dy/dx - (x+y) = 1 - x^2 =0 ; y(0) =1 
\]

\begin{Shaded}
\begin{Highlighting}[]
\NormalTok{f<-}\ControlFlowTok{function}\NormalTok{(fcn,x,y)\{}
  \KeywordTok{return}\NormalTok{(}\KeywordTok{eval}\NormalTok{(fcn))}
\NormalTok{\}}

\NormalTok{obtenerErrorAbsoluto<-}\ControlFlowTok{function}\NormalTok{(x,y)\{}
\NormalTok{  solucion=}\KeywordTok{exp}\NormalTok{(x)}\OperatorTok{*}\NormalTok{((}\OperatorTok{-}\NormalTok{x}\OperatorTok{*}\KeywordTok{exp}\NormalTok{(}\OperatorTok{-}\NormalTok{x))}\OperatorTok{-}\KeywordTok{exp}\NormalTok{(}\OperatorTok{-}\NormalTok{x)}\OperatorTok{+}\DecValTok{2}\NormalTok{)}
  \KeywordTok{return}\NormalTok{(}\KeywordTok{abs}\NormalTok{(y}\OperatorTok{-}\NormalTok{solucion))}
\NormalTok{\}}

\NormalTok{graficarCampoPendiente<-}\ControlFlowTok{function}\NormalTok{(x0, xn, y0, yn, fcn, numpendientes, metodo)\{}
\NormalTok{  apma1 <-}\StringTok{ }\ControlFlowTok{function}\NormalTok{(t, y, parameters)\{}
\NormalTok{    a <-}\StringTok{ }\NormalTok{parameters[}\DecValTok{1}\NormalTok{] }
\NormalTok{    dy <-}\StringTok{ }\NormalTok{a}\OperatorTok{*}\NormalTok{(}\KeywordTok{f}\NormalTok{(fcn, t, y))}
    \KeywordTok{list}\NormalTok{(dy)}
\NormalTok{  \} }
\NormalTok{  apma1.flowField <-}\StringTok{ }\KeywordTok{flowField}\NormalTok{(apma1, }\DataTypeTok{x =} \KeywordTok{c}\NormalTok{(x0, xn), }
                               \DataTypeTok{y   =} \KeywordTok{c}\NormalTok{(y0, yn), }\DataTypeTok{parameters =} \KeywordTok{c}\NormalTok{(}\DecValTok{1}\NormalTok{), }
                               \DataTypeTok{points =}\NormalTok{ numpendientes, }\DataTypeTok{system =} \StringTok{"one.dim"}\NormalTok{, }
                               \DataTypeTok{add =} \OtherTok{FALSE}\NormalTok{, }\DataTypeTok{xlab =} \StringTok{"x"}\NormalTok{, }\DataTypeTok{ylab =} \StringTok{"y"}\NormalTok{, }
                               \DataTypeTok{main =}\NormalTok{ metodo)}
  \KeywordTok{grid}\NormalTok{()}
\NormalTok{\}}

\NormalTok{graficarSolucionNumerica<-}\ControlFlowTok{function}\NormalTok{ (x, y)\{}
  \KeywordTok{points}\NormalTok{ (x, y, }\DataTypeTok{pch=}\DecValTok{20}\NormalTok{, }\DataTypeTok{col=}\StringTok{"blue"}\NormalTok{)}
  \ControlFlowTok{for}\NormalTok{ (i }\ControlFlowTok{in} \DecValTok{2}\OperatorTok{:}\KeywordTok{length}\NormalTok{(x))\{}
    \KeywordTok{segments}\NormalTok{(x[i}\OperatorTok{-}\DecValTok{1}\NormalTok{], y[i}\OperatorTok{-}\DecValTok{1}\NormalTok{], x[i], y[i], }\DataTypeTok{col=}\StringTok{"red"}\NormalTok{)}
\NormalTok{  \}}
\NormalTok{\}}

\NormalTok{Rrk4<-}\ControlFlowTok{function}\NormalTok{(dy, ti, tf, y0, h, }\DataTypeTok{graficar=}\OtherTok{TRUE}\NormalTok{, }\DataTypeTok{numpendientes=}\DecValTok{10}\NormalTok{)\{}
\NormalTok{  t<-}\KeywordTok{seq}\NormalTok{(ti, tf, h)}
\NormalTok{  y<-}\KeywordTok{c}\NormalTok{(y0)}
  \KeywordTok{cat}\NormalTok{(}\StringTok{"x    |y         |k1        |k2        |k3        |k4       |error absoluto}\CharTok{\textbackslash{}n}\StringTok{"}\NormalTok{)}
  \ControlFlowTok{for}\NormalTok{(i }\ControlFlowTok{in} \DecValTok{2}\OperatorTok{:}\KeywordTok{length}\NormalTok{(t))\{}
\NormalTok{    k1=h}\OperatorTok{*}\KeywordTok{f}\NormalTok{(dy, t[i}\OperatorTok{-}\DecValTok{1}\NormalTok{], y[i}\OperatorTok{-}\DecValTok{1}\NormalTok{])}
\NormalTok{    k2=h}\OperatorTok{*}\KeywordTok{f}\NormalTok{(dy, t[i}\OperatorTok{-}\DecValTok{1}\NormalTok{]}\OperatorTok{+}\NormalTok{h}\OperatorTok{/}\DecValTok{2}\NormalTok{, y[i}\OperatorTok{-}\DecValTok{1}\NormalTok{]}\OperatorTok{+}\NormalTok{k1}\OperatorTok{*}\NormalTok{(}\FloatTok{0.5}\NormalTok{))}
\NormalTok{    k3=h}\OperatorTok{*}\KeywordTok{f}\NormalTok{(dy, t[i}\OperatorTok{-}\DecValTok{1}\NormalTok{]}\OperatorTok{+}\NormalTok{h}\OperatorTok{/}\DecValTok{2}\NormalTok{, y[i}\OperatorTok{-}\DecValTok{1}\NormalTok{]}\OperatorTok{+}\NormalTok{k2}\OperatorTok{*}\NormalTok{(}\FloatTok{0.5}\NormalTok{))}
\NormalTok{    k4=h}\OperatorTok{*}\KeywordTok{f}\NormalTok{(dy, t[i}\OperatorTok{-}\DecValTok{1}\NormalTok{]}\OperatorTok{+}\NormalTok{h, y[i}\OperatorTok{-}\DecValTok{1}\NormalTok{]}\OperatorTok{+}\NormalTok{k3)}
\NormalTok{    y<-}\KeywordTok{c}\NormalTok{(y, y[i}\OperatorTok{-}\DecValTok{1}\NormalTok{]}\OperatorTok{+}\DecValTok{1}\OperatorTok{/}\DecValTok{6}\OperatorTok{*}\NormalTok{(k1}\OperatorTok{+}\DecValTok{2}\OperatorTok{*}\NormalTok{k2}\OperatorTok{+}\DecValTok{2}\OperatorTok{*}\NormalTok{k3}\OperatorTok{+}\NormalTok{k4))}
    \KeywordTok{cat}\NormalTok{(t[i}\OperatorTok{-}\DecValTok{1}\NormalTok{],}\StringTok{" | "}\NormalTok{, y[i}\OperatorTok{-}\DecValTok{1}\NormalTok{],}\StringTok{" | "}\NormalTok{,k1,}\StringTok{" | "}\NormalTok{,k2,}\StringTok{" | "}\NormalTok{,k3,}\StringTok{" | "}\NormalTok{,k4,}\StringTok{" | "}\NormalTok{,}\KeywordTok{obtenerErrorAbsoluto}\NormalTok{(t[i}\OperatorTok{-}\DecValTok{1}\NormalTok{],y[i}\OperatorTok{-}\DecValTok{1}\NormalTok{]),}\StringTok{"}\CharTok{\textbackslash{}n}\StringTok{"}\NormalTok{)}
\NormalTok{  \}}
  \ControlFlowTok{if}\NormalTok{ (graficar)\{}
    \KeywordTok{graficarCampoPendiente}\NormalTok{(}\KeywordTok{min}\NormalTok{(t), }\KeywordTok{max}\NormalTok{(t), }\KeywordTok{min}\NormalTok{(y), }\KeywordTok{max}\NormalTok{(y), dy, numpendientes, }\StringTok{"RK4"}\NormalTok{)}
    \KeywordTok{graficarSolucionNumerica}\NormalTok{(t, y)}
\NormalTok{  \}}
\NormalTok{  rta<-}\KeywordTok{list}\NormalTok{(}\DataTypeTok{w=}\NormalTok{y, }\DataTypeTok{t=}\NormalTok{t)}
\NormalTok{\}}

\NormalTok{rk3<-}\ControlFlowTok{function}\NormalTok{(dy, ti, tf, y0, h, }\DataTypeTok{graficar=}\OtherTok{TRUE}\NormalTok{, }\DataTypeTok{numpendientes=}\DecValTok{10}\NormalTok{)\{}
\NormalTok{  t<-}\KeywordTok{seq}\NormalTok{(ti, tf, h)}
\NormalTok{  y<-}\KeywordTok{c}\NormalTok{(y0)}
  \KeywordTok{cat}\NormalTok{(}\StringTok{"x    |y         |k1         |k2        |k3       |error absoluto}\CharTok{\textbackslash{}n}\StringTok{"}\NormalTok{)}
  \ControlFlowTok{for}\NormalTok{(i }\ControlFlowTok{in} \DecValTok{2}\OperatorTok{:}\KeywordTok{length}\NormalTok{(t))\{}
\NormalTok{    k1=h}\OperatorTok{*}\KeywordTok{f}\NormalTok{(dy, t[i}\OperatorTok{-}\DecValTok{1}\NormalTok{], y[i}\OperatorTok{-}\DecValTok{1}\NormalTok{])}
\NormalTok{    k2=h}\OperatorTok{*}\KeywordTok{f}\NormalTok{(dy, t[i}\OperatorTok{-}\DecValTok{1}\NormalTok{]}\OperatorTok{+}\NormalTok{h}\OperatorTok{/}\DecValTok{2}\NormalTok{, y[i}\OperatorTok{-}\DecValTok{1}\NormalTok{]}\OperatorTok{+}\NormalTok{k1}\OperatorTok{*}\NormalTok{(}\FloatTok{0.5}\NormalTok{))}
\NormalTok{    k3=h}\OperatorTok{*}\KeywordTok{f}\NormalTok{(dy, t[i}\OperatorTok{-}\DecValTok{1}\NormalTok{]}\OperatorTok{+}\NormalTok{h, y[i}\OperatorTok{-}\DecValTok{1}\NormalTok{]}\OperatorTok{-}\NormalTok{k1}\OperatorTok{+}\DecValTok{2}\OperatorTok{*}\NormalTok{k2)}
\NormalTok{    y<-}\KeywordTok{c}\NormalTok{(y, y[i}\OperatorTok{-}\DecValTok{1}\NormalTok{]}\OperatorTok{+}\DecValTok{1}\OperatorTok{/}\DecValTok{6}\OperatorTok{*}\NormalTok{(k1}\OperatorTok{+}\DecValTok{4}\OperatorTok{*}\NormalTok{k2}\OperatorTok{+}\NormalTok{k3))}
    \KeywordTok{cat}\NormalTok{(t[i}\OperatorTok{-}\DecValTok{1}\NormalTok{],}\StringTok{" | "}\NormalTok{, y[i}\OperatorTok{-}\DecValTok{1}\NormalTok{],}\StringTok{" | "}\NormalTok{,k1,}\StringTok{" | "}\NormalTok{,k2,}\StringTok{" | "}\NormalTok{,k3,}\StringTok{" | "}\NormalTok{,}\KeywordTok{obtenerErrorAbsoluto}\NormalTok{(t[i}\OperatorTok{-}\DecValTok{1}\NormalTok{],y[i}\OperatorTok{-}\DecValTok{1}\NormalTok{]),}\StringTok{"}\CharTok{\textbackslash{}n}\StringTok{"}\NormalTok{)}
\NormalTok{  \}}
  \ControlFlowTok{if}\NormalTok{ (graficar)\{}
    \KeywordTok{graficarCampoPendiente}\NormalTok{(}\KeywordTok{min}\NormalTok{(t), }\KeywordTok{max}\NormalTok{(t), }\KeywordTok{min}\NormalTok{(y), }\KeywordTok{max}\NormalTok{(y), dy, numpendientes, }\StringTok{"RK3"}\NormalTok{)}
    \KeywordTok{graficarSolucionNumerica}\NormalTok{(t, y)}
\NormalTok{  \}}
\NormalTok{  rta<-}\KeywordTok{list}\NormalTok{(}\DataTypeTok{w=}\NormalTok{y, }\DataTypeTok{t=}\NormalTok{t)}
\NormalTok{\}}

\NormalTok{r<-}\KeywordTok{Rrk4}\NormalTok{(}\KeywordTok{expression}\NormalTok{(x}\OperatorTok{+}\NormalTok{y}\OperatorTok{+}\DecValTok{1}\OperatorTok{-}\NormalTok{x}\OperatorTok{^}\DecValTok{2}\NormalTok{), }\DecValTok{0}\NormalTok{, }\DecValTok{1}\NormalTok{, }\DecValTok{1}\NormalTok{, }\FloatTok{0.1}\NormalTok{)}
\end{Highlighting}
\end{Shaded}

\begin{verbatim}
## x    |y         |k1        |k2        |k3        |k4       |error absoluto
## 0  |  1  |  0.2  |  0.21475  |  0.2154875  |  0.2305488  |  0 
## 0.1  |  1.215171  |  0.2305171  |  0.2457929  |  0.2465567  |  0.2621727  |  0.1048288 
## 0.2  |  1.461402  |  0.2621402  |  0.2779972  |  0.2787901  |  0.2950192  |  0.2185966 
## 0.3  |  1.739858  |  0.2949858  |  0.3114851  |  0.31231  |  0.3292168  |  0.3401402 
## 0.4  |  2.051823  |  0.3291823  |  0.3463914  |  0.3472519  |  0.3649075  |  0.4681739 
## 0.5  |  2.398719  |  0.3648719  |  0.3828655  |  0.3837652  |  0.4022485  |  0.6012768 
## 0.6  |  2.782116  |  0.4022116  |  0.4210722  |  0.4220152  |  0.4414132  |  0.7378787 
## 0.7  |  3.20375  |  0.441375  |  0.4611937  |  0.4621846  |  0.4825934  |  0.8762442 
## 0.8  |  3.665537  |  0.4825537  |  0.5034314  |  0.5044753  |  0.5260012  |  1.014455 
## 0.9  |  4.169599  |  0.5259599  |  0.5480078  |  0.5491102  |  0.5718709  |  1.150392
\end{verbatim}

\includegraphics{DocProyecto_files/figure-latex/unnamed-chunk-6-1.pdf}

\begin{Shaded}
\begin{Highlighting}[]
\NormalTok{r2<-}\KeywordTok{rk3}\NormalTok{(}\KeywordTok{expression}\NormalTok{(x}\OperatorTok{+}\NormalTok{y}\OperatorTok{+}\DecValTok{1}\OperatorTok{-}\NormalTok{x}\OperatorTok{^}\DecValTok{2}\NormalTok{), }\DecValTok{0}\NormalTok{, }\DecValTok{1}\NormalTok{, }\DecValTok{1}\NormalTok{, }\FloatTok{0.1}\NormalTok{)}
\end{Highlighting}
\end{Shaded}

\begin{verbatim}
## x    |y         |k1         |k2        |k3       |error absoluto
## 0  |  1  |  0.2  |  0.21475  |  0.23195  |  0 
## 0.1  |  1.215158  |  0.2305158  |  0.2457916  |  0.2636226  |  0.1048165 
## 0.2  |  1.461376  |  0.2621376  |  0.2779945  |  0.2965227  |  0.2185703 
## 0.3  |  1.739816  |  0.2949816  |  0.3114806  |  0.3307795  |  0.3400979 
## 0.4  |  2.051763  |  0.3291763  |  0.3463851  |  0.3665357  |  0.4681134 
## 0.5  |  2.398638  |  0.3648638  |  0.382857  |  0.4039488  |  0.6011956 
## 0.6  |  2.782012  |  0.4022012  |  0.4210612  |  0.4431933  |  0.737774 
## 0.7  |  3.203618  |  0.4413618  |  0.4611799  |  0.4844616  |  0.8761127 
## 0.8  |  3.665375  |  0.4825375  |  0.5034144  |  0.5279667  |  1.014293 
## 0.9  |  4.169402  |  0.5259402  |  0.5479872  |  0.5739437  |  1.150196
\end{verbatim}

\includegraphics{DocProyecto_files/figure-latex/unnamed-chunk-6-2.pdf}


\end{document}
